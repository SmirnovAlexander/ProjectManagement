\documentclass[a4paper,8pt]{article}

% Encoding.
\usepackage{geometry}
\usepackage[T2A]{fontenc}
\usepackage[utf8]{inputenc}
\usepackage[english,russian]{babel}

% Code insertion.
\usepackage[outputdir=build]{minted}

% Math functions.
\usepackage{amsmath}

% Image insertion.
\usepackage{svg}

% No line breaks.
\usepackage[none]{hyphenat}

\title{Задание 8: План собеседования в проект ``e-гриб'' \\ на позицию Middle Backend Python Developer}
\author{
    \begin{tabular}[t]{c@{\extracolsep{8em}}c} 
        Афанасов Артём     & Смирнов Александр \\
        &\\ 
        Струтовский Максим & Федор Жилкин
    \end{tabular}
}

\date{\today}

\begin{document}

\maketitle

\section*{Резюме}

Кандидаты отбираются путем конкурса портфолио. Кандидаты должны соответствовать всем поставленным требованиям вакансии. Затем им необходимо выполнить тестовое задание и пройти личное собеседование.

\section*{Удаленное собеседование с HR-менеджером}

\begin{itemize}
    \item Рассказ о компании;
    \item Рассказ о проектах с открытой вакансией;
    \item Просьба кандидату рассказать о себе;
    \item Задать вопрос: "Почему именно наша компания?";
    \item Задать вопрос: "Как Вы видите свое будущее в компании?";
    \item Ответы на другие вопросы кандидата.
\end{itemize}

\section*{Тестовое задание}

Кандидатам, прошедшим конкурс портфолио, предлагается решить тестовое задание. Разработать python-проект "Файловый менеджер" с использованием микросервисной архитектуры. В рамкам тествого задания кандидат должен показать умение применять архитектурные паттерны (приложение обязательно должно соотвествовать требованием микросервисной архитектуры), реализовывать GUI, уверенное владение python и библиотеками.  

\subsection*{Задача}

Необходимо разработать файловый менеджер и выложить docker-образ на Docker Hub.

\subsubsection*{Обязательные требования}

\begin{itemize}
    \item просмотр директорий;
    \item открытие файлов;
    \item копирование, вырезка, переименование, удаление файлов;
    \item рекурсивный поиск по папкам и файлам;
    \item архивация файлов и директорий;
    \item GUI со свободным оформлением и выбором фреймворков.
\end{itemize}

\subsubsection*{Опциональные требования}

\begin{itemize}
    \item просмотр содержимого архивов без их распаковки;
    \item поиск файлов по регулярным выражениям.
\end{itemize}

\section*{Личное собеседование с техническим специалистом}

Кандидаты, успешно справившиеся с тестовым заданием приглашаются на собеседование.

\subsection*{Вводная часть}

Примерное время 7-10 минут. Перед собеседованием подготовить распечатанное портфолио кандидата для интервьюеров. 

\begin{itemize}
    \item В начале собеседования Small-talk и оценка soft-skills кандидата.
    \item Рассказ о задачах команды и целях компании.
    \item Просьба кандидата рассказать о себе.
    \item Сообщить кандидату схему собеседования.
\end{itemize}

\subsection*{Обсуждение предыдущего опыта}

Примерное время 10 минут.

\begin{itemize}
    \item Просьба кандидату рассказать об одном ключевом проекте. 
    \begin{itemize}
        \item Особое внимание уделить архитектурной составляющей проекта. 
        \item Описание трудностей в проекте и способов их решения.
    \end{itemize}
\end{itemize}

\subsection*{Обсуждение тестового задания}

Примерное время 10-15 минут.

\begin{itemize}
    \item Примененные технологии;
    \item Примененные паттерны;
    \item Оценка чистоты и качества кода;
    \item Обсуждение упущенной функциональности, важной для проекта;
    \item Обратная связь от кандидата о задаче.
\end{itemize}

\subsection*{Оценка знаний}

Примерное время 30-40 минут.

\begin{itemize}
    \item Фундаментальные знания (около 10 минут):
        \begin{itemize}
            \item Вопросы на знание алгоритмов, структур данных, компьютерных сетей, архитектурных шаблонов, архитектурных видов.
        \end{itemize}
    \item Прикладные знания (около 15 минут):
        \begin{itemize}
            \item Применение паттернов проектирования.
            \item Тестирование на уровне разработки.
            \item Знание языка python.
        \end{itemize}
    \item Инструментальные знание (около 5 минут):
        \begin{itemize}
            \item CI и использование сопутствующих инструментов (TeamCity, Jenkins, AppVeyor).
            \item Смоделировать проблемную ситуацию с VCS Git и попросить озвучить решение.
        \end{itemize}
    \item Дополнительные вопросы согласно резюме кандидата.
\end{itemize}

\subsection*{Организационная часть}

\begin{itemize}
    \item Причина ухода с предыдущего места работы.
    \item Условия работы: график, заработная плата, формат работы (удаленно или очно).
    \item Обсуждение времени испытательного срока.
    \item Готовность приступить к работе в нашей компании.
\end{itemize}

\subsection*{Ответы на вопросы кандидата}

Примерное время 5-15 минут.

\section*{По окончании собеседований}

\begin{itemize}
    \item Обсуждение и ранжирование кандидатов.
    \item Отправка писем кандидатам с результатами.
\end{itemize}

\end{document}
