\documentclass[a4paper,8pt]{article}

% Encoding.
\usepackage{geometry}
\usepackage[T2A]{fontenc}
\usepackage[utf8]{inputenc}
\usepackage[english,russian]{babel}

% Code insertion.
\usepackage[outputdir=build]{minted}

% Math functions.
\usepackage{amsmath}

% Image insertion.
\usepackage{svg}

% No line breaks.
\usepackage[none]{hyphenat}

\title{Задание 7: Вакансия на проект ``e-гриб''}
\author{
    \begin{tabular}[t]{c@{\extracolsep{8em}}c} 
        Афанасов Артём     & Смирнов Александр \\
        &\\ 
        Струтовский Максим & Феодор Жилкин
    \end{tabular}
}

\date{\today}

\begin{document}

\maketitle

\section*{Описание вакансии}

\subsection*{Middle Backend Python Developer}

    \begin{itemize}
        \item Требуемый опыт работы: 1 -- 3 лет;
        \item Полная занятость, полный день;
        \item От 150 000 руб.
    \end{itemize}

``e-гриб'' --- это продуктовая экосистема, направленная на покупку и обслуживание чайного гриба. Наш сервис позволяет приобрести комплект для выращивания чайного гриба и получить консультации для наиболее эффективного выращивания грибного компаньона.

Сейчас мы расширяем команду и ищем \textbf{Middle Backend Python Developer} для участия в создании и продвижении нового продукта, которым будут пользоваться тысячи клиентов по всей России. Развивай продукт, которым уже пользуются сотни людей.

\subsubsection*{Чем предстоит заниматься}

    \begin{itemize}
        \item Принятие архитектурных решений;
        \item Интеграция со сторонними сервисами;
        \item Написание сервисной логики;
        \item Реализация микросервисов для рекомендательных систем.
    \end{itemize}

\subsubsection*{Чего ждем}

    \begin{itemize}
        \item Опыт работы с *nix-системами;
        \item Опыт разработки микросервисов и их интеграции с внешними сервисами, опыт работы с Git;
        \item Опыт работы с фреймворком Flask/Asyncio. Владение библиотеками: numpy, math, celery. Не обязательно ограничиваться данным списком. Опыт написания API-приложений (REST API, gRPC).
    \end{itemize}

\subsubsection*{Будет плюсом}

    \begin{itemize}

        \item Умение писать и тестировать инфраструктурный код, понимание концепции Infrastructure as Code;
        \item Понимание принципов, опыт построения отказоустойчивых сервисов и эксплуатации высоконагруженных систем (web-серверов, реляционных БД, серверов приложений).;
        \item Понимание архитектуры, принципов и механизмов работы ОС Linux (управление процессами, файловые системы, сетевой стек) на уровне системного администратора;
        \item Опыт работы с основными системами оркестрации и контеризации (отличное знание одной из них).
    \end{itemize}


\subsubsection*{Нашим сотрудникам мы предлагаем}


    \begin{itemize}
        \item Job-offer обсуждается индивидуально в зависимости от уровня имеющихся компетенций и навыков;
        \item Оформление по ТК РФ;
        \item Частичная компенскация фитнеса;
        \item ДМС;
        \item Гибкий рабочий график (работа в офисе, начало рабочего дня до 11 ч);
        \item Работу в быстрорастущей и динамичной компании;
        \item Амбициозный проект, который уже заслужил признание аудитории;
        \item Использование новейших технологий;
        \item Адекватный лояльный менеджмент;
        \item Прозрачную систему профессионального роста и развития в компании;
        \item Развитую культуру менторства;
        \item Яркую корпоративную жизнь;
        \item Комфортный офис в центре города (станция метро ``Чёрная речка''), БЦ класса А+ с кухней и комнатой отдыха;
        \item Полноценное корпоративное питание.
    \end{itemize}

\subsubsection*{Контактная информация}
    \begin{itemize}
        \item Служба персонала: job@egrib.ru:
            \begin{itemize}
                \item Тема письма: backend4;
            \end{itemize}
        \item Телефон: 8-(800)-555-35-35.
    \end{itemize}

\textit{``e-гриб''! Для настоящих профи своего дела мы открыты круглосуточно!}


\newpage
\section*{План распространения вакансии}

Вакансия будет распространяться с использованием следущих платформ и технологий:

\subsection*{Внутри компании}
    \begin{itemize}
        \item Знакомые уже существующих сотрудников;
        \item Премии за успешных кандидатов;
    \end{itemize}

\subsection*{HeadHunter}

    \begin{itemize}
        \item Предложенная выше вакансия будет размещена на сайте hh.ru;
    \end{itemize}

\subsection*{LinkedIn}

    \begin{itemize}
        \item Будем искать людей, у которых в статусе стоит ``поиск работы'' и предлагать им нашу вакансию;
    \end{itemize}

\subsection*{Сайты объявлений}

    \begin{itemize}
        \item Например, avito;
    \end{itemize}

\subsection*{Собственный сайт вакансии}

    \begin{itemize}
        \item Можно продвигать таким образом собственный бренд;
    \end{itemize}

\subsection*{Тематические сообщества}

    \begin{itemize}
        \item Разместить объявления о вакансии в соответсвующих тематических пабликах и сообществах;
        \item Предложить вакансию участникам профильных конференций;
    \end{itemize}

\subsection*{Контекстная реклама}

    \begin{itemize}
        \item Размещение ссылок на вакансии.
    \end{itemize}

\end{document}
