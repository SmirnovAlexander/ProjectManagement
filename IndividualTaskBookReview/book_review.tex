
\documentclass[a4paper,8pt]{article}

% Encoding.
\usepackage{hyperref}
\usepackage{geometry}
\usepackage[T2A]{fontenc}
\usepackage[utf8]{inputenc}
\usepackage[english,russian]{babel}

% Code insertion.
\usepackage[outputdir=build]{minted}

% Math functions.
\usepackage{amsmath}

% Image insertion.
\usepackage{svg}

% No line breaks.
\usepackage[none]{hyphenat}

\title{Индивидуальное задание 2: Отчёт о книге}
\author{ 
    \begin{tabular}[t]{c@{\extracolsep{8em}}c}
        Смирнов Александр     & Феодор Жилкин \\
    \end{tabular}
}

\date{\today}

\begin{document}

\maketitle
\tableofcontents
\newpage



\section{Описание книги}

``Бизнес с нуля. Метод Lean Startup для быстрого тестирования идей и выбора бизнес-модели'' -- книга американского предпринимателя Эрика Риса, в которой объединяются принципы, подходы и практики таких концепций как бережливое производство, развитие клиентов и гибкая методология разработки.

Бережливый стартап — это концепция создания компаний, разработки и выведения на рынок новых продуктов и услуг, основанная на таких понятиях, как научный подход к менеджменту стартапов, подтвержденное обучение, проведение экспериментов, итеративный выпуск продуктов для сокращения цикла разработки, измерение прогресса, и получение ценной обратной связи от клиентов. Используя этот подход, компании могут проектировать продукты и услуги, которые бы соответствовали ожиданиям и потребностям клиентов без необходимости большого объёма первичного финансирования или затратных продуктовых запусков.

\section{Что понравилось и почему}

Книга понравилась тем, что заставляет задуматься о подходе к разработке стартапа.
Начинаешь понимать, что уже на ранних этапах развития проекта необходимо больше откликов от клиента.
Очень важной частью является изучение клиентов и получение обратной связи по продукту.
Необходимо непрерывно проводить эксперименты и проверять, нужен ли клиенту тот продукт, который мы делаем.

Также важно найти ранних последователей, которые были бы лояльны к продукту.

Не нужно останавливаться при отсутствии успеха, необходимо вернуться в начало и попытаться понять, действительно ли у людей есть проблема, которую мы пытаемся решить, или мы её сами придумали.

\section{Что не понравилось и почему}

Книгу можно ужать до 20-30 страниц, очень много ненужной воды, мысли повторяются по несколько раз. Также не хватает структурированности. Хотелось бы видеть книгу, состоящую из глав: инсайт + примеры. В таком случае не нужны бесконечные прелюдии, так как в примерах будет сразу отображено влияние и польза приведённого инсайта.


\section{Что будете использовать в своем проекте (2-3 наиболее интересных идеи)}

В своём проекте однозначно будем непрерывно собирать обратную связь на каждой итерации создания продукта. Также будем проводить исследования по изучению целевой аудитории нашего продукта, чтобы иметь подтверждение тому, что продукт создаётся не в вакууме, а действительно кому-то необходим.


\section{План внедрения новых знаний}

Разработка стартапа начнётся с исследования целевой аудитории. Будет составлен проект ЦА, и в фокус-группах исследована потребность в продукте. Также будут проведены исследования с использованием контекстной рекламы таким образом, будто продукт уже существует (для оценки количество лидов). Затем при дальнейшей разработке продукта будут постоянно проводиться исследования по поводу необходимости того или иного функционала в приложении.


\section{Критерий оценки эффективности внедрения}

Непосредственными критериями эффективности будет служить объективная оценка ЦА в продукте. Не возникнет ситуации, при которой команда будет разрабатывать что-то в течение продолжительного времени а затем поймёт, что это никому не нужно.

Таким образом, критерием для оценки эффективности будет служить количество лидов, вовлечённость пользователей в продукт, количество обратной связи.


\end{document}
