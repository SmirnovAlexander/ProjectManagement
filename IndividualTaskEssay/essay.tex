
\documentclass[a4paper,8pt]{article}

% Encoding.
\usepackage{hyperref}
\usepackage{geometry}
\usepackage[T2A]{fontenc}
\usepackage[utf8]{inputenc}
\usepackage[english,russian]{babel}

% Code insertion.
\usepackage[outputdir=build]{minted}

% Math functions.
\usepackage{amsmath}

% Image insertion.
\usepackage{svg}

% No line breaks.
\usepackage[none]{hyphenat}

\title{Индивидуальное задание 3: Эссе}
\author{ Смирнов Александр }

\date{\today}
\begin{document}

\maketitle
% \tableofcontents
% \newpage


\section{Почему я хочу стать руководителем проекта}

Ещё до прохождения курса мне хотелось попробовать себя в роли РП\footnote{Руководитель проекта}. Было не совсем понятно, что включают в себя задачи РП, однако ближе к концу курса начало появляться осознание задач РП. Нужно учитывать великое множество факторов, чтобы не провалить проект. Особенно это стало заметно после групповых домашних заданий, в которых нужно подробнейшим образом расписывать огромное количество деталей. Мне показалось интересным вместе с командой обсуждать идею (тему) проекта, составлять планы, продумывать бюджет и ограничения. 

Работа в команде помогла мне почуствовать себя РП, так как было необходимо:

    \begin{itemize}
        \item Обсуждать свои и чужие идеи;
        \item Прислушиваться к другим точкам зрения;
        \item Находить компромиссы;
        \item Планировать задачи разделять их между собой.
    \end{itemize}

В ближайшем будущем я обязательно попробую себя в качестве РП с компаний единомышленников открыть свое дело или запустить стартап. И буду пробовать себя не только потому, что это просто нравится, а потому еще, что узнал в общих чертах необходимую информацию для старта и сопровождения проекта. 

Хотелось бы отметить, что управлять проектом -- нелёгкая задача, нужно уметь думать а перспективах, уметь правильно планировать и, самое главное, уметь грамотно координировать людей для достижения поставленных целей.

\end{document}
