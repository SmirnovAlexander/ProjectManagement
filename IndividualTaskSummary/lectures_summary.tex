
\documentclass[a4paper,8pt]{article}

% Encoding.
\usepackage{hyperref}
\usepackage{geometry}
\usepackage[T2A]{fontenc}
\usepackage[utf8]{inputenc}
\usepackage[english,russian]{babel}

% Code insertion.
\usepackage[outputdir=build]{minted}

% Math functions.
\usepackage{amsmath}

% Image insertion.
\usepackage{svg}

% No line breaks.
\usepackage[none]{hyphenat}

\title{Индивидуальное задание 1: Конспект лекций}
\author{ Смирнов Александр }

\date{\today}
\begin{document}

\maketitle
\tableofcontents
\newpage


\section{Конспект от 16.02.2021}

\subsection{Введение}

\subsubsection{Проект}


    \begin{itemize}
        \item Ограниченное время;
        \item Сроки;
        \item Уникальность;
        \item Цель;
        \item Нетривиальность.
    \end{itemize}


\subsubsection{Успех проекта}

    \begin{itemize}
        \item Выполнение проекта согласно ограничениям.
    \end{itemize}

\subsubsection{Операционная деятельность }

    \begin{itemize}
        \item Постоянство;
        \item Непрерывность;
        \item Длительность;
        \item Повторяемость.
    \end{itemize}

\subsubsection{Примеры}

    \begin{itemize}
        \item Разработка сайта;
        \item Сопровождение бухгалтерской системы;
        \item Изучение английского языка;
        \item Получение водительского удостоверения;
        \item Обучение в универе;
    \end{itemize}

\subsection{Специфика проектов разработки ПО}


    \begin{itemize}
        \item ПО незримо;
        \item Работа имеет нелинейную сложность;
        \item Возникают проблемы, например, вопрос масштабируемости;
        \item Производительность сотрудников различается в разы;
        \item Требования меняются по ходу проекта.
    \end{itemize}

\subsection{``Участники'' проекта}

    \begin{itemize}
        \item Менеджер проекта;
        \item Команда проекта;
        \item Стейкхолдеры.
    \end{itemize}


\subsection{Задачи менеджера проекта}


    \begin{itemize}
        \item Обеспечить успех проекта;
        \item Коммуникация с заказчиком и командой;
        \item Нужен баланс между выполнением своими руками и делегированием задач;
        \item Планирование и контроль;
        \item Управление;
    \end{itemize}


\subsection{Область знаний}

    \begin{itemize}
        \item Социальная психология;
        \item Финансовый менеджмент и бухгалтерия;
        \item Право;
        \item Закупки и цепочки поставок;
        \item Административное управление;
    \end{itemize}



\subsection{Виды проектов}


    \begin{itemize}
        \item Внутренние;
        \item Инвестиционные;
        \item Заказные;
    \end{itemize}


\subsection{Модель оплаты}

\subsubsection{Fixed Price}

    \begin{itemize}
        \item Фиксированные деньги;
        \item Риски удорожания лежат на подрядчике;
        \item Управление проектами лежит на подрядчике;
        \item Приемка по окончании;
        \item Заказчик может неоправданно переплатить за риск;
        \item Подрядчик может перевыполнить задания сверх нормы без прибыли;
        \item Подрядчик может заработать больше, если умеет хорошо оценивать.
    \end{itemize}


\subsubsection{Time \& Materials}

    \begin{itemize}
        \item Заказчик оплачивает во время выполнения;
        \item Заказчик управляет проектом;
        \item Заказчик формирует команду;
        \item Заказчик отчетность по часам и затратам;
        \item Выгоднее по деньгам для заказчика, если есть доверия и добросовестность с обеих сторон;
        \item Гибкое внесение изменений по мере работы.
    \end{itemize}


\newpage
\section{Конспект от 16.03.2021}




\subsection{Технико-коммерческое предложение}

    \begin{itemize}
        \item Информация о компании;
        \item Рамки проекта (обычно описывается заказчиком);
        \item Описание решения проекта;
        \item Процедуры управления проектом, предлагаемым заказчиком;
        \item Команда;
        \item Структурная декомпозиция работ -- WBS (Word Breakdown Structure);
        \item Оценка проекта;
        \item Условия оплаты.
    \end{itemize}


\subsection{Оценка проекта}


    \begin{itemize}
        \item Время;
        \item Деньги;
    \end{itemize}

    \subsubsection{Сложности оценивания}

        \begin{itemize}
            \item Нет опыта;
            \item Требования не финализированы -- неопределённость;
            \item Большой объём проекта;
            \item Кто выполняет?
            \item Нет знаний по предметной области и технологиям;
            \item Нефункциональные требования;
            \item Конус неоределённости;
            \item Вывод: невозможно сразу дать истинную оценку.
        \end{itemize}


    \subsubsection{Методы оценки}


        \begin{itemize}
            \item Оценка по аналогу;
            \item Нужные аналогичные проекты;
            \item Параметризация -- быстро прикинуть, не особо погружаясь;
            \item Экспертная оценка -- можно задействовать несколько экспертов или совместить с PERT;
            \item Planning poker -- спрашиваем всё мнение команды;
            \item UCP -- Use Case Points;
            \item PERT = (O + 4R + P) / 6 -- совокупность нескольких оценок на основе разных предположений;
            \item Wideband Delphi -- как покер, только для глобальных задач;
            \item Есть много возможностей оценить, но нужны ресуры;
            \item Если проводить оценку регулярно, точность будет повышаться.
        \end{itemize}

    \subsubsection{Что ошибочно пропускают при оценке}


     
        \begin{itemize}
            \item Разработка требований;
            \item Разработку системы развёртывания и само развёртывание;
            \item Документация;
            \item Внедрение в процессы заказчика;
            \item Миграция данных;
            \item Интеграция со сторонними системами;
            \item Обновление на новую версию ``на лету'' с сохранением данных;
            \item Интеграционное и системное тестирование;
            \item Передача кода во владение заказчика.

        \end{itemize}





\newpage
\section{Конспект от 30.03.2021}

\subsection{Оценка срока}

\subsubsection{Типовые пропуски в оценке}


    \begin{itemize}
        \item Управление персоналом;
        \item Организационные и общекомандные мероприятия;
        \item Отпуска, праздники, больничные;
        \item Время на адаптацию и обучение сотрудников;
        \item Стоимость оборудования и ПО.
    \end{itemize}

\subsubsection{Ответственные за оценку}

    \begin{itemize}
        \item Тех. эксперт;
        \item Менеджер (руководитель);
        \item Вместе они также отвечают за оценку времени + можно перераспределять обязанности между ними.
    \end{itemize}


\subsubsection{Создание календарного графика}

    \begin{itemize}
        \item Взять задачи из WBS;
        \item Оценить объём и трудоёмкость каждой;
        \item Оценить время задач;
        \item Найти зависимости между задачами (FS, FF, SS, SF);
        \item Проверить ограничения по времени;
        \item Проверить баланс ресурсов;
        \item Наметить основные вехи.
    \end{itemize}

\subsubsection{Рекомендации}


    \begin{itemize}
        \item Прописывать вехи;
        \item Прописывать зависимости (1-2 вида);
        \item Помнить о праздниках: своих и заказчика;
        \item Расписывать для каждой задачи все требуемые ресурсы, не только 1 главный;
        \item Проверить загрузку ресурсов;
        \item Декомпозировать задачи, чтобы занимали <= 5 дней;
        \item Описывать выдачу каждой задачи;
        \item Проверять критический путь и зависимости;
        \item Расписать основные даты;
        \item Выписывать все задачи, в том числе и управленческие.
    \end{itemize}


\subsection{Оценка стоимости}



\subsubsection{Модель оплаты Fixed Price}

    \begin{itemize}
        \item Подрядчик обязуется выполнить за фикс. стоимость;
        \item Риски и управление на нём;
        \item Качество / сроки могут упасть;
        \item У заказчика нет доступа к деталям управления;
        \item Приёмка по окончании;
        \item Плюсы с точки зрения заказчика;
        \item Fix price;
    \end{itemize}


\subsubsection{Модель оплаты Time \& Resources}


    \begin{itemize}
        \item Оплачиваются ресурсы;
        \item Отчётность по часам и затратам + по этапам или месяцам;
        \item Заказчик может принимать участие в управлении проектом и формировании команд;
    \end{itemize}


\subsubsection{Комбинированные модели оплаты}

    \begin{itemize}
        \item Time \& Resources с порогом стоимости, за который нельзя выходить.
    \end{itemize}


\subsubsection{Оценка Fixed-price}

    \begin{itemize}
        \item Себестоимость (стоимость ресурсов -- обычно часы*ставка, в ставке учтены налоги, аренда и т.д., + специальные расходы типа hardware, soft) + вознаграждение за риск (30-50\% от стоимости работ);
        \item Оплата по фазам, после приёмки;
        \item Большая часть оплаты после завершения;
        \item Оговариваются случаи, в которых нужно вернуть деньги;
        \item Оговаривают, что происходит, если контракт разрывается.
    \end{itemize}

\subsubsection{Оценка Time \& Resources:}

    \begin{itemize}

        \item Цена = кол-во человек * ставка/час * кол-во часов;
        \item Оплата по месяцам/этапам;
        \item Оплата доп расходов.
    \end{itemize}



\subsection{Управление персоналом}

    \begin{itemize}
        \item Кадровый учёт (формальная деятельность -- оформить всех по ГОСТу);
        \item Подбор сотрудников (и интеграция);
        \item Формирование команды;
        \item Обучение;
        \item Аттестация и оценка (проверка квалифицированности);
        \item Мотивация;
        \item Увольнение.
    \end{itemize}


\end{document}
